	\subsubsection{Recon3D}
	
	Recon3D represents the most comprehensive human metabolic network model to date \cite{Brunk2018}. The model,Recon3DModel\_301.mat, was downloaded from http://vmh.life. The original biomass equation was modified to adjust the biomass demand reported by \cite{Niklas2013} for the parental line AGE1\_HN. All the other original demands from recon3D were deactivated. An extra reaction representing the maintenance demand, not associated with growth, of atp was set according to \cite{Fernandez-de-Cossio-Diaz2018b}. All the fluxes representing exchangeable metabolites (external reactions) were set as reversible (lb and ub set to a large number), so the only effective bound constraints (for FBA and EP) are the ones produced by the chemostat consideration \cite{Fernandez-de-Cossio-Diaz2018b}. Additionally, to include the molecular crowding constraints we map \cite{Shlomi2011} enzymatic costs, initially defined for Recon1, to Recon3D.
	
	Pursuing to reproduce the conditions of the different steady states, the external concentrations of the exchangeable metabolites were set using the data from $Tables\ 1-2$. Additionally, external concentrations of salts, oxygen, and other metabolites not specified in the medium were set to a large number. Particularly, Recon3D was unable to grow without pe\_hs[e], phosphatidylethanolamine, (or similar) lipid in the feed medium. Later we will discuss the impact of this metabolite in the medium.  