\subsection{Experimental data} 
	Experimental data were taken from \cite{Rath2017a}. In this work, the author performed 6 continuous cultures, ($A$, $B$, $C$, $D$, $E$, $F$), with the cell line AGE1.HN.AA1. The parental line AGE1.HN was established by the company ProBioGen (ProBioGen AG, Berlin, Germany) from a tissue sample of a human brain. All culture's feed mediums were based on the standard 42-Max-UB-medium, which is serum-free and was specially developed for the AGE1.HN cell line $Table\ 1$. The experiments were run under various conditions, differing mainly in the dilution rate ($D$) and the feed medium composition of glucose ($GLC$), glutamine ($GLN$) and galactose ($GAL/GALC$) $Table\ 2$.
	
	For each experiment, a steady-state condition was reached, ($A$, $B$, $C$, $D$, $E$, $F01$), and several observables  was reported $Tables\ 3-4$. Particularly relevant for this work was the growth rate ($\mu$), $D$, the viable cell density ($Xv$) and the medium concentration ($s$) and derived uptake rate ($u$) for a set of metabolites ($GLC$, lactose ($LAC$), $GLN$, ammonium ($NH4$), $GAL$, pyruvate ($PYR$), glutamate ($GLU$), alanine ($ALA$), asparagine ($ASP$)). A unit conversion was required to make experimental data and our model compatible. For this propose the only external data needed was the cell mass density. It was used  0.25 pgDW/ $\mu$$m^3$ \cite{Niklas2011}.