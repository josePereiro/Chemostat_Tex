\subsection{Low concentrations of phosphatidylethanolamine} 
	
	Because phosphatidylethanolamine isn't a reported component of the 42-Max-UB-medium, $Tabla\ 1$, and it can be a carbon source for the GEMs, its concentration was set first to the lowest value possible. For Recon3D it was fixed to 0.1mM, a handpicked value, and for CHO it wasn't present in the feed medium at all.
	
	Flux balance analysis with molecular crowding (FBA) was performed similarly to \cite{Fernandez-de-Cossio-Diaz2018b}. Plots of $\mu$ and $Xv$ as a function of $\xi$ are shown in $Figure\ 1$. Both networks failed to reach the experimental values for both observables. In the case of $Xv$, in all the explored $\xi$ range (from 1 to 1000 gDW hr/ L) the value was underestimated. This result may have several causes. One could be that the GEM have a "too expensive" biomass equation. We mean that one or more biomass required precursors might be overestimated. However, this reason seems hard to sustain because we modified Recon3D biomass equation with the reported anabolic biomass demand for AGE1.HN.AA1 \cite{Niklas2013}. We use the reported anabolic demand of proteins, lipids, DNA, RNA and carbohydrates to fix the total demand of this same groups in the original Recon3D biomass equation and the changes wasn't too big. For CHO we kept the biomass equation unmodified. Nonetheless, a modification in the biomass equation would be an elegant way to solve the problem. Other plausible cause could be a big difference between the models frameworks. But, both works model the chemostat in an equivalent fashion, as can be seen when comparing equations 1 with 3.25 and 2 with 3.28 and 3.29 from \cite{Fernandez-de-Cossio-Diaz2017} and \cite{Rath2017a} respectively.
	 
	Moreover, other parameter than can affect $Xv$ is the bleeding coefficient, $\phi$,
	defined in perfusion systems as the fraction of cells that escape from the culture through a cell-retention device \cite{Fernandez-de-Cossio-Diaz2017}. Because, \cite{Rath2017a} doesn't report the use of any retention devise in the cultures this parameter was taken as 1.0, but lower values will cause to increment the predicted value of  $Xv$ by the model. Finally, it needs to be taken into consideration that we are using GEMs that are not directed curated for the working cell line. This is, by far, the most difficult factor to solve, because the curing of a genome scale metabolic network is a very laborious labor.

	
	On other hand, correlations of experimental and modeled uptakes, $Figure\ 2$, was performed. The graphs shows better results for the uptakes of $GLC$ and $GLN$, 
	and worse correlations for uptakes such as $PYR$, $NH4$ and $LAC$ for both GEMs. In general, CHO had better correlations than Recon3D. This could be, maybe, explained because Recon3D is a general GEM, allowing the model to access to all the possible genome repertory present in humans, a fact that is not accurate for a cell in the $in\ vivo$ scenario. Additionally, CHO was curated for a defined cell line, representing a more restricted and realistic network. Anyway, it is remarkable that the models reproduce uptakes like $GLC$ and $GLN$ regarding all the previous consideration.